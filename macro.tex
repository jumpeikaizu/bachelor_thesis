%-*- coding: utf-8 -*-

% ToDo や,各自のコメント行の表示のon/offは,次の命令で切り替える.
\newif\ifdraft \drafttrue   %% 作業用
%\newif\ifdraft\draftfalse  %% 提出用

%%%%%%%%%%%%%%%%%%%% コメント用 %%%%%%%%%%%%%%%%%%

\ifdraft
% 自分の好きな色を選んで下さい
\newcommand{\shinocom}[1]{\textcolor{magenta}{[(篠原)#1]}}
\newcommand{\naricom}[1]{\textcolor{red}{[(成澤)#1]}}
\newcommand{\usercom}[1]{\textcolor{blue}{[(user)#1]}}
\newcommand{\todo}[1]{{\color{red}{[ToDo: #1]}}}
\else
\newcommand{\shinocom}[1]{}
\newcommand{\naricom}[1]{}
\newcommand{\usercom}[1]{}
\newcommand{\todo}[1]{}
\fi

%%%%%%%%%%%%%%%%%%%% パッケージ %%%%%%%%%%%%%%%%%%

\usepackage[dvipdfmx]{color}                      % 色の使用
\usepackage[dvipdfmx]{graphicx}                % 図表
\usepackage{theorem}                           % 【必須】証明のフォントを変える
\usepackage{amssymb}                           % 特殊文字が使用可能となる
\usepackage{amsmath}                           % math中の\text
% \usepackage{comment}                           % 複数行コメントアウト
% \usepackage{multirow}                          % 表の縦結合
% \usepackage{subfigure}                         % 図の中にサブ(a)(b)とか作るやつ
% \usepackage{colortbl}                          % 表の網掛け
% \usepackage{graphicx}                          % 文字列の縦圧縮
\usepackage{url}                               % URLを使うとき
\usepackage[ruled, linesnumbered]{algorithm2e} % アルゴリズム4.0

\usepackage{okumacro} %ルビ
%\usepackage{pxrubrica} %ルビ

%%%%%%%%%%%%%%%%%%%% マクロ定義 %%%%%%%%%%%%%%%%%%

%QED : 全角文字なのが気になる人は自分で調べてね
\newcommand{\qed}{\hfill \rmfamily{□}} 
% argmaxを追加
\newcommand{\argmax}{\mathop{\rm arg~max}\limits}
% \newcommand{}{}
% \newcommand{}[1]{#1}


%%%%%%%%%%%%%%%%%%% 定理環境定義 %%%%%%%%%%%%%%%%%%

%%%%%%%%%%%% 基本的にここから下はいじらない%%%%%%%%%%%%ここから

\theoremstyle{plain}
\theorembodyfont{\normalfont}
\setlength\theorempreskipamount{14pt}
\setlength\theorempostskipamount{14pt}

\newtheorem{definition}{定義}
\newtheorem{theorem}{定理}
\newtheorem{lemma}{補題}
\newtheorem{example}{例}
\newtheorem{proof}{証明}
\newtheorem{proposition}{命題}
\newtheorem{observation}{観察}
\newtheorem{problem}{問題}
\newtheorem{corollary}{系}
\newtheorem{remark}{注}
\newtheorem{fact}{事実}
\renewcommand{\theproof}{}

%%%%%%%%%%%% 基本的にここはいじらない%%%%%%%%%%%%ここまで

