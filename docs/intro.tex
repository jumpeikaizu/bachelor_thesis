%-*- coding: utf-8 -*-
\section{背景}
\label{background}
近年,様々なゲームに対して人工知能の研究が行われている.ゲームにはすべての情報がプレイヤに公開されている完全情報ゲームと,プレイヤに公開されていない情報がある不完全情報ゲームがある.チェス,オセロ,将棋といった二人完全情報ゲームでは,人間のトッププレイヤと同等,またはそれ以上の実力をもった人工知能が作成されている~\cite{othello,deep,syogi}.
それに対して,ポーカーや麻雀などの多人数不完全情報ゲームは,あまり研究が進んでいない.\if0 人間より強い人工知能の作成は.\fi 不確定である情報の推定や複数の他プレイヤの状況を考慮する必要があり,人工知能の設計が複雑になるためである.そのため,人間のトッププレイヤ以上の実力をもった人工知能の作成は難しい~\cite{poker1,poker2}.
特に麻雀は不確定である情報や考慮しなければならない要素が多いため,人間のトッププレイヤ以上の実力をもった人工知能は我々の知る限り作成されていない~\cite{majang3,majang2,majang4,majang1}.

本論文では,\if0 麻雀を題材にしているが前述したようにAI作成が難しいため,\fi 麻雀の多人数性を排除した,一人麻雀を研究の対象とする.一人麻雀では,他プレイヤの状況の考慮や他プレイヤの不確定情報の推定をする必要がないため,人工知能の設計が簡単になる,多人数性を排除してはいるが,一人麻雀の人工知能は多人数の麻雀の人工知能に拡張するための土台となる.一人麻雀において,上がり率や上がり点の高さ,上がりの早さなどを測定をすることで,多人数の麻雀における効率の良い行動選択が,どのくらいできているかを見ることができる.\if0 一人麻雀での効率の良い行動選択は,多人数の麻雀において重要である.\fi また,多人数の麻雀では最終的に一番多くの点数をもっていたプレイヤの勝利となるため,上がり率は重要である.ゲーム中,自分が最下位のときは上がり点の高さが重要であり,自分がトップのときはゲームを早く終了させるために早上がりが重要となる.

一人麻雀を対象とした研究は以前から行われている~\cite{majang2,majang4},そのなかでも,モンテカルロ法~\cite{montecarlo}を基にした手法がある~\cite{che,komatu}.\if0 小松らはモンテカルロ法を適用した効率的な手法を提案している.\fi モンテカルロ法とは,ある行動に対してプレイアウトと呼ばれるシミュレーションを行い,その行動を評価する手法である.プレイアウト時の不確定な情報やプレイヤの行動はランダムに決定される.モンテカルロ法は盤面の状態に対する評価を必要としないため,評価関数の設定が難しい不完全情報ゲームに対して有効であると考えられる.しかし,一人麻雀においては,そのままモンテカルロ法を適用すると,ほとんどのプレイアウトで報酬が得られず,各行動の良さを評価しにくい.モンテカルロ法を基にした手法の一つである小松らの手法は,この問題を改善している.小松らの手法では,報酬が得られやすい効率的なプレイアウトを行い,一人麻雀において上がり点が高くなる打ち方をする.\if0 人工知能の作成ができている\fi 

しかし,小松らの手法は早く上がること,上がり率を高くすることを考えていない.実際の多人数で行う麻雀では,成績が良いプレイヤほど,平均上がり点が低く上がり率が高いことが統計で明らかになっている~\cite{kagaku}.また,多人数であれば自分と他プレイヤの状況を考慮し,上がり点を高くする打ち方や,早く上がる打ち方に変える必要がある.よって,本研究は打ち方を変えることができる手法を提案する.

\section{本論文の内容}
多人数の麻雀においては,ゲーム終了時に最も点数を持っていたプレイヤの勝ちとなる.そのため,ゲーム中自分が最下位のときは上がり点を高くする打ち方にすべきであり,自分がトップのときはゲームを早く終了させるために早上がりをする打ち方にすべきである.このように,多人数の麻雀では状況に応じた打ち方の変更が重要である.
そのため,本論文では,一つのパラメータを変更することで打ち方を変えることができる手法を提案する.小松らの手法における,プレイアウト時の報酬の値を変更する.提案手法は,上がり点を高くするための評価指標と早上がりをするための評価指標の二つを報酬として用いる.また,一人麻雀の実験を行い,提案手法のパラメータを変更することで打ち方が変わるか確かめる.実験において早上がりを目指す貪欲な手法,上がり点を高くすることを目指す小松らの手法を指標とし比較を行う.提案手法は,実際の多人数に重要な打ち方の変更が可能なため,多人数の麻雀に適用しやすい手法であると考えられる.\if0 提案手法では,前節で述べた多人数の麻雀に必要な打ち方の変更を実現できる.\fi

\section{構成}
第2章では,一般的な麻雀と一人麻雀のルールついて説明する.第3章では,早上がりを目指す貪欲な手法,小松らの手法を説明する.第4章では提案手法を説明する.第5章では実験から説明した手法を比較する.第6章では,まとめと今後の課題について述べる.
%スラスラはだめ 
%,が多い文は2文に