%-*- coding: utf-8 -*-
\section{実験}
実験により,提案手法の$\alpha$を変更することで打ち方が変わるか確かめる.実験環境を以下に示す.
\begin{itemize}
\item OS:CentOS 6.2
\item CPU:Intel Xeon CPU X5660 @ 2.80GHz
\item コア数:12
\item メモリ:48GB
\item HDD:2TB
%\item グラフィックボード:nVidia Corporation GT218 [GeForce210] 
\end{itemize}

1万回ゲームを行い,早上がりを目指す貪欲な手法,小松らの手法,提案手法を上がり率,ゲーム回数による平均点数,上がり回数による平均点数,平均上がり巡目の面から比較する.山は同じデータセットであり,小松らの手法,提案手法のプレイアウトは1000回とする.提案手法の$\alpha$は0から1まで0.1刻みで行う.また,ヒューマンプレイヤとして本論文の著者も,一人麻雀のゲームを100回行った.

結果を表~\ref{skpcomparison}に示す.提案手法の$\alpha=1.0$と小松らの手法は実質的に同様なので同じ結果としている.$\alpha$の値を変えることにより,平均上がり巡目と上がり回数による平均点数が変化していることがわかる.$\alpha$が小さいときは早上がりを目指すようになっているため上がり率も小松らの手法に比べ良くなっている.また,高得点と早上がりのバランスがとれているためか,$\alpha$によってはゲーム回数による平均点が小松らの手法を上回っている.一般的な麻雀においては最終的に最も点数を持っているプレイヤの勝ちなので優位な結果と言える.$\alpha$が大きいときは,上がり回数による平均点はヒューマンプレイヤに勝っている.上がり率は貪欲な手法には負けるが,$\alpha$の値を変えることで,一般的な麻雀に必要な打ち方を変えることができる手法にすることができた.


\begin{table}[t]
	\caption{貪欲な手法,小松らの手法,提案手法,ヒューマンプレイヤの比較}
	\label{skpcomparison}
	\begin{center}
	 \begin{tabular}{|c|r|r|r|r|}
	 	\hline
	 	&             & \multicolumn{1}{c|}{ゲーム全体の} & \multicolumn{1}{c|}{上がったときの} & \\
	 	& 上がり率(\%) & \multicolumn{1}{c|}{平均点数}     & \multicolumn{1}{c|}{平均点数}     & 平均上がり巡目 \\ \hline
	 	貪欲な手法           & 18.91 & 739.0 & 3908.1 & 14.06 \\ \hline 
	 	モンテカルロ法        & 7.02 	& 352.1 & 5015.8 & 14.50 \\ \hline
	 	提案手法$\alpha=0.0$ & 17.51 & 693.0 & 3957.6 & 14.02 \\ \hline
	 	    $\alpha=0.1$ & 16.99 & 828.6 & 4877.0 & 13.95 \\ \hline
	 	    $\alpha=0.2$ & 17.31 & 865.3 & 4998.8 & 14.02 \\ \hline
	 	    $\alpha=0.3$ & 16.88 & 863.0 & 5112.3 & 14.09 \\ \hline
	 	    $\alpha=0.4$ & 16.34 & 851.3 & 5210.0 & 14.14 \\ \hline
	 	    $\alpha=0.5$ & 15.71 & 849.8 & 5409.5 & 14.21 \\ \hline
		    $\alpha=0.6$ & 15.66 & 904.5 & 5775.7 & 14.36 \\ \hline
		    $\alpha=0.7$ & 14.69 & 904.8 & 6159.6 & 14.39 \\ \hline
		    $\alpha=0.8$ & 14.00 & 911.4 & 6510.1 & 14.58 \\ \hline
		    $\alpha=0.9$ & 13.16 & 931.5 & 7078.2 & 14.68 \\ \hline
		    $\alpha=1.0$ & & & & \\
			   (小松らの手法) & 11.27 & 904.1 & 8022.4 & 14.90 \\ \hline
		ヒューマンプレイヤ     & 24.00 & 1329.0 & 5537.5 & 12.88 \\ \hline
	 \end{tabular}
	\end{center}
\end{table}