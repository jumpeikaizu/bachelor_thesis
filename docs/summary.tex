%-*- coding: utf-8 -*-
本論文では,小松らの手法の報酬を変え,一つのパラメータを変更するだけで柔軟に打ち方を変えることができる手法を提案した.一般的な多人数の麻雀のルールを簡略化した一人麻雀による実験で,提案手法が確かに打ち方を変えることができる手法であることを示した.これにより,他プレイヤの状況を考慮しなければならない実際の麻雀に必要な打ち方の変更を実現できた.

今後の課題として,本論文では,一人麻雀を扱ったが,多人数の麻雀への拡張を目指す.素朴な手法などの人工知能を他プレイヤとした対戦や,人工知能の接続を認めている麻雀サーバを用いて人間との対戦を行い,評価を行いたい.また,その場合には,他プレイヤを考慮した際の動作を拡張する必要がある.具体的には,他プレイヤが捨てた牌で上がる,他プレイヤが捨てた牌をもらう鳴き,上がることを諦める降りという行為を,他プレイヤと自分の状況を考慮して行うことが必要である.麻雀は最終的な点数で勝敗が決まるため,拡張としてゲーム全体,他プレイヤ,自分の状況から,その状況に最適な打ち方に自動で変えるということなども考えられる.